\chapter{sheet-12 : The Path Integral Formulation of Quantum Theory}
\section{Background Materials}
\begin{enumerate}
	\item Basis states \\
	\begin{align}
		\hat{Q}_s \ket{q} &= q\ket{q} \\
		\hat{P}_s \ket{p} &= p\ket{p} 
	\end{align}
	The states $\{\ket{q}\}$ and $\{\ket{p}\}$ are basis states, i.e., $\mathbbm{1}$
	\begin{align}
		\int \dd{q} \ket{q}\bra{q} &= \mathbb{1} \\
		\int \dd{p} \ket{p}\bra{p} &= \mathbb{\hat{1}} \\
	\end{align}
	where the normalization is chosen as
	\begin{align}
		\braket{q}{q^\prime} &= \delta(q-q^\prime) \\
		\braket{p}{p^\prime} &= \delta(p-p^\prime) 
	\end{align}
	The operators $\hat{Q}_s$ and $\hat{P}_s$ can be expressed in coordinate representation as follows
	\begin{align}
		\bra{q}\hat{Q}_s &= q \bra{q} \\
		\bra{q}\hat{P}_s &= -\iu \hbar \pdv{q} \bra{q} 
	\end{align}
	In momentum representaion
	\begin{align}
		\bra{p}\hat{Q}_s &= \iu \hbar \pdv{p} \bra{p} \\
		\bra{p}\hat{P}_s &=  \bra{p} 
	\end{align}
	The fundamental commutation relation between $\hat{Q}$ and $\hat{P}$ is 
	\begin{equation}
		\left[\hat{Q}_s, \hat{P}_s\right] = \iu
		 \hbar \mathbb{\hat{1}}
	\end{equation}
	For later purposes we will need the momentum eigenstates in coordinate representation, i.e., $\braket{q}{p}$. To find $\braket{q}{p}$ we proceed as follows
	\begin{align}
		\hat{P}_s \ket{p} &= p\ket{p} \\
		\bra{q}\hat{P}_s \ket{p} &= p\bra{q}\ket{p} \\
		-\iu \hbar \pdv{q} \braket{q}{p} &= p \braket{q}{p}
	\end{align}
	This equation is easy to solve for $\braket{q}{p}$. We find
	\begin{equation}
		\braket{q}{p} = C e^{\iu p q / \hbar}
	\end{equation}
	The constant $C$ is chosen such that we have the normalization $\braket{p}{p^\prime } \delta(p-p^\prime)$. Now
	\begin{align*}
		\braket{p}{p^\prime} 
		&= \int \dd{q} \braket{p}{q}\braket{q}{p^\prime} \\
		&= \int \dd{q} C^* e^{-ipq/\hbar} C^{\iu p^\prime q /\hbar} \\
		&= \left|C\right|^2 \int_{-\infty}^{\infty} \dd{q} e^{-\iu(p-p^\prime)q/\hbar} \\
		&= \left|C\right|^2  2\pi \delta(\frac{p-p^\prime}{\hbar}) \\
		&= \left|C\right|^2  2\pi \hbar \delta(p-p^\prime)
	\end{align*}
	Choosing $C$ to be real and positive, we must have
	\begin{equation}
		C = \frac{1}{\sqrt{2\pi \hbar}}
	\end{equation}
	Thus
	\begin{equation}
		\braket{q}{p} = \frac{1}{\sqrt{2\pi \hbar}} e^{\iu p q / \hbar}
	\end{equation}
	
	
	\item Quantum Mechanics in Heisenberg picture//
	The Heisenberg picture of quantum dynamics is obtained from the Schr\"{o}dinger picture by the following transformation of all kets and all operators
	\begin{align}
		\ket{}_H &= e^{\iu \hat{H}t/\hbar}\ket{}_s \label{chapter12.eqn-schrodinger-to-heisengerg-ket}\\
		\hat{\Omega}_H(t) &= e^{\iu \hat{H}t/\hbar} \hat{\Omega}_S e^{-\iu \hat{H}t/\hbar}
		\label{chapter12.eqn-schrodinger-to-heisengerg-operator}
	\end{align}
	where we have assumed the system is conservative, i.e., $\hat{H}$ is independent of time. \\
	
	In the Heisenberg picture, the base kets, for example, the eigenkets of $\hat{Q}_H(t)$ and $\hat{P}_H(t)$ are time dependent. We have
	\begin{align}
		\hat{Q}_H(t) \ket{q,t}_H &= q \ket{q,t}_H \\
		\hat{P}_H(t) \ket{p,t}_H &= p \ket{p,t}_H
	\end{align}
	where
	\begin{align}
		\ket{q,t}_H &= e^{\iu \hat{H}t/\hbar}\ket{q}\\
		\ket{p,t}_H &= e^{\iu \hat{H}t/\hbar}\ket{p}
	\end{align}
	and
	\begin{align}
		\hat{Q}_H(t) &= e^{\iu \hat{H}t/\hbar} \hat{Q}_S e^{-\iu \hat{H}t/\hbar}\\
		\hat{P}_H(t) &= e^{\iu \hat{H}t/\hbar} \hat{P}_S e^{-\iu \hat{H}t/\hbar}
	\end{align}
	The orthogonality and completeness of the Heisenberg picture base kets are
	\begin{align}
		_H\braket{q,t}{q^\prime,t}_H = \braket{q}{q^\prime} &=  \delta(q-q^\prime) \\
		_H\braket{p,t}{p^\prime,t}_H = \braket{p}{p^\prime} &=  \delta(p-p^\prime) 
	\end{align}
	Note that these are equal time relations. And
	\begin{align}
		\hat{\mathbb{1}} &= \int \dd{q} \ket{q,t}_H \  {_H}\bra{q,t}\label{chapter12.eqn1-completeness-q}\\
		\hat{\mathbb{1}} &= \int \dd{p} \ket{p,t}_H\ {_H}\bra{p,t}\label{chapter12.eqn1-completeness-p}
	\end{align}
	To show the validity of equation (\ref{chapter12.eqn1-completeness-q}), for example, we use the transformation of kets and bras from the schr\"{o}dinger picture to the Heisenberg picture (\ref{chapter12.eqn-schrodinger-to-heisengerg-ket}), i.e., 
	\begin{equation}
		{_H}\bra{} = {_S}\bra{} e^{-\iu \hat{H}t/\hbar} 
	\end{equation}
	Thus the right hand side of equation (\ref{chapter12.eqn1-completeness-q}) can be written as
	\begin{align}
		\int \dd{q}  \ket{q,t}_H {_H}\bra{q,t} &= \int \dd{q} \quad e^{\iu \hat{H}t/\hbar}\ket{q} \bra{q}e^{-\iu \hat{H}t/\hbar} \\
		&= e^{\iu \hat{H}t/\hbar}\left(\int \dd{q}  \ket{q} \bra{q}\right)e^{-\iu \hat{H}t/\hbar} \\
		&= e^{\iu \hat{H}t/\hbar}\quad\mathbb{1}\quad e^{-\iu \hat{H}t/\hbar} \\
		&= \mathbb{1}
	\end{align}
	We note that the \underline{state vector} in the Heisenberg picture is independent of time, while the state vector in the Schr\"{o}dinger picture is time-dependent. This is very simply shown as follows:
	\begin{align}
		\ket{\psi}_H &= e^{\iu \hat{H}t/\hbar} \ket{\psi(t)}_S \\
		&= e^{\iu \hat{H}t/\hbar} e^{-\iu \hat{H}t/\hbar} \ket{\psi(0)}_S\\
		&= \ket{\psi(0)}_S
	\end{align}
	Thus the state ket in the Heisenberg picture is independent of time and is same as the initial state ket in the Schr\"{o}dinger picture.
	
	
	\section{Propagator}
	The dynamics of a quantum system is completely specified by the "Feynman Kernel", or the propagator or the transition amplitude defined as
	%%%%%%%%%%%%%
	%%%%%%%%%% TODO
	%%%%%%%%%%%%%
	\begin{equation}
		U(q_2, t_2; q_1, t_1) = {\_H}\braket{q_2, t_2}{q_1, t_1}_H
		\label{chapter12.eqn1-propagator}
	\end{equation}
	Transforming to the Schr\"{o}dinger picture base kets, we can write equation (\ref{chapter12.eqn1-propagator}) as
	\begin{align}
		U(q_2, t_2; q_1, t_2) 
		&= \bra{q_2}e^{-\iu \hat{H}t_2/\hbar} e^{\iu \hat{H}t_1/\hbar} \ket{q_1} \nonumber \\
		&= \bra{q_2}e^{-\iu \hat{H}(t_2-t_1)/\hbar} \ket{q_1} \label{chapter12.eqn2-propagator}
	\end{align}
	We see that the propagator is the matrix element in the coordinate basis of the time-evolution operator in the Schr\"{o}dinger picture. The physical meaning of the propagator is that it is the probability amplitude of finding the particle at $q_2$ at time $t_2$ if the particle was at $q_1$ at an earlier time $t_1$. Knowing the propagator is equivalent to solving the Schr\"{o}dinger equation, for it allows us to calculate the Schr\"{o}dinger picture wave function at any moment of time if the wave function is known at an earlier moment. This is shown below:
	\begin{align}
		\psi_S(q,t )
		&= \braket{q}{\psi_S(t)} \\
		&= \bra{q}e^{-\iu \hat{H}t/\hbar} \ket{\psi_S(0)} \\
		&= {\_H}\braket{q,t}{\psi}_H \\
		&= \int \dd{q^\prime} {\_H}\braket{q,t}{q^\prime, t^\prime}_H {\_H}\braket{q^\prime, t^\prime}{\psi}_H \\
		&= \int \dd{q^\prime} U(q,t;q^\prime, t^\prime) \psi_S(q^\prime,t^\prime)
	\end{align}
	The path integral formalism of quantum dynamics provides a means to construct the transition amplitude ${\_H}\braket{q,t}{q^\prime, t^\prime}_H$ from the classical Hamiltonian or Lagrangian alone, without any reference to non commuting operators or Hilbert space vectors.\\
	
	\section{Path Integral for the Propagator}
		We will now calculate
		\begin{equation}
			U(x,t;x_0, t_0) = {\_H}\braket{x,t}{x_0, t_0}
			\label{chapter12.eqn1-propagator-path-integral}
		\end{equation}
		where $t>t_0$. For this purpose let us devide the time integral $(t,t_0)$ into $N$ equal segments each of duration $\epsilon$. Namely, let
		\begin{equation}
			\epsilon = \frac{t-t_0}{N}
			\label{chapter12.eqn2-propagator-path-integral}
		\end{equation}
		In other words, we are discretizing the time interval, and in the end we will take the continuum limit $\epsilon\rightarrow 0$ and $N\rightarrow \infty$. We label the end times $t_0$ and $t$ and the intermediate times as $t_1, t_2, \ldots, t_{N-1}, t_N=t$. Further, we will let $x_N=x$. The intermediate times are 
		\begin{equation}
			t_i=t_0+i \epsilon, \quad i=1,2,\ldots,N-1
			\label{chapter12.eqn3-propagator-path-integral}
		\end{equation}
		
		At each intermediate time a complete set of basis states $\ket{x_i,t_i}$ may be inserted:
		\begin{equation}
			\braket{x t}{x_0 t_0} =\int_{-\infty}^{\infty}\dd{x_1} \ldots \int_{-\infty}^{\infty} \dd{x_{N-1}} \braket{x t}{x_{N-1} t_{N-1}}\braket{x_{N-1} t_{N-1}}{x_{N-2} t_{N-2}} \ldots \braket{x_2 t_2}{x_{1} t_{1}} \braket{x_1 t_1}{x_{0} t_{0}}
			\label{chapter12.eqn4-propagator-path-integral}
		\end{equation}
		
		Here we have omitted the subscript $H$ in the Heisenberg picture basis vectors since there is no scope for confusion. Note that while there are $N$ scalar products in euquation (\ref{chapter12.eqn4-propagator-path-integral}), there are only $N-1$ intermediate points so that the number of integrations is $N-1$. Since $x_N = x$ and $t_N= t$, we can write equation (\ref{chapter12.eqn4-propagator-path-integral}) as
		\begin{equation}
			\braket{x t}{x_0 t_0} = \int \prod_{i=1}^{N-1} \dd{q_i} \prod_{i=0}^{N-1} \braket{x_{i+1} t_{i+1}}{x_i t_i}
			\label{chapter12.eqn5-propagator-path-integral}
		\end{equation}
		Equation (\ref{chapter12.eqn5-propagator-path-integral}) can be interpreted as follows: A particle that propagates from $x_0$ at time $t_0$ to $x$ at time $t$ can take an arbitrary intermediate trajectory, shown in figure (\ref{chapter12.fig1-propagator-path-integral-trajectory})
		\begin{figure}
			\centering
			\caption{text}
			\label{chapter12.fig1-propagator-path-integral-trajectory}
		\end{figure}
	Such a path is characterized by the coordinate values $x_i$ at intermediate grid @@@@ in the time interval $(t_0,t)$. One such path is shown in the figure as a zigzag curve. Since each intermediate coordinates $x_i\quad (i=1,2,\ldots,N-1)$ can vary form $-\infty$ to $\infty$, it is essential that all conceivable paths connecting the end points are taken into account. According tot the representation principle of Quantum Mechanics they all contribute to the transition amplitude (\ref{chapter12.eqn5-propagator-path-integral}). Of course, some trajectories may turn out to be more important than others.\\
	
	We will now calculate the intermediate scalar products which themselves are propagators but over infinitesimal time intervals. An intermediate scalar product has the form $\braket{x_{i+1} t_{i+1}}{x_i t_i}$. We can calculate this inner product up to first order in $\epsilon$ from equation (\ref{chapter12.eqn2-propagator-path-integral}) as follows
	\begin{align}
		\braket{x_{i+1} t_{i+1}}{x_i t_i} &= \bra{x_{i+1}} e^{-\iu \hat{H}t_{i+1}/\hbar}e^{\iu \hat{H}t_{i}/\hbar}\ket{x_i}\\
		&= \bra{x_{i+1}} e^{-\iu \hat{H}(t_{i+1} - t_i)/\hbar}\ket{x_i}\\
		&= \bra{x_{i+1}} e^{-\iu \hat{H}\epsilon/\hbar}\ket{x_i}\\
		&= \bra{x_{i+1}} \left(\mathbb{1} - \frac{\iu \epsilon}{\hbar}\hat{H} + \order{\epsilon^2}\right)\ket{x_i}
	\end{align}
	We will take $\hat{H}$ to be of the form
	\begin{equation}
		\hat{H} = \frac{\hat{P}}{2 m} + V(\hat{X})
	\end{equation}
	Therefore
	\begin{align}
		\braket{x_{i+1} t_{i+1}}{x_i t_i} 
		&= \bra{x_{i+1}}\left[\mathbb{1} - \frac{\iu \epsilon}{\hbar}\left(\frac{\hat{P}^2}{2 m} + V(\hat{X}) + \right) + \order{\epsilon^2}\right]\ket{x_i}\\
		&= \int_{-\infty}^{\infty} \dd{p} \braket{x_{i+1}}{p} \bra{p} \left[\mathbb{1} - \frac{\iu \epsilon}{\hbar}\left(\frac{\hat{P}^2}{2 m} + V(\hat{X})\right) + \order{\epsilon^2}\right] \ket{x_i} \\
		&= \int_{-\infty}^{\infty} \dd{p} \braket{x_{i+1}}{p} \braket{p}{x_i} \left[\mathbb{1} - \frac{\iu \epsilon}{\hbar}\left(\frac{p^2}{2 m} + V(x_i)\right) + \order{\epsilon^2}\right]  \\
		&= \int \dd{p} \frac{1}{\sqrt{2\pi \hbar}}e^{\iu p x_{i+1}/\hbar}\frac{1}{\sqrt{2\pi \hbar}}e^{-\iu p x_{i}/\hbar} e^{-\iu \epsilon/\hbar \left( \frac{p^2}{2 m} + V(x_i) \right)} \\
		&= \frac{1}{2\pi\hbar} \int_{-\infty}^{\infty} \dd{p} e^{\iu p (x_{i+1} - x_i)/\hbar} e^{-\iu \epsilon/\hbar \left( \frac{p^2}{2 m} + V(x_i) \right)} \\
		&= \frac{1}{2\pi\hbar} e^{-\iu \epsilon V(x_i)/\hbar} \int_{-\infty}^{\infty} e^{\iu p \epsilon\frac{(x_{i+1} - x_i)}{\epsilon}\frac{1}{\hbar}} e^{-\iu \epsilon/\hbar \left( \frac{p^2}{2 m}\right)} \dd{p}\\
		&= \frac{1}{2\pi\hbar} e^{-\iu \epsilon V(x_i)/\hbar} \int_{-\infty}^{\infty} e^{\iu p \epsilon \dot{x}_i / \hbar} e^{-\iu \epsilon \frac{p^2}{2 m \hbar}} \dd{p} \\
		&= \frac{1}{2\pi\hbar} e^{-\iu \epsilon V(x_i)/\hbar} \int_{-\infty}^{\infty} e^{-\frac{\iu \epsilon}{2 m \hbar} \left(p^2 - 2 m p \dot{x}_i \right)} \dd{p}
		\label{chapter12.eqn6-path-integral-propagator}
	\end{align}
	In the above we have defined
	\begin{equation}
		\dot{x}_i = \frac{x_{i+1} x_i}{\epsilon}
	\end{equation}
	Now
	\begin{equation}
		p^2 - 2m p \dot{x}_i = (p- m \dot{x}_i)^2 - m^2 \dot{x}_i^2
	\end{equation}
	We make the change of variable
	\begin{equation}
		p^\prime = p -m \dot{x}_i
	\end{equation}
	Therefore equation (\ref{chapter12.eqn6-path-integral-propagator}) can be written as
	
	\begin{align}
		\braket{x_{i+1} t_{i+1}}{x_i t_i} 
		&= \frac{1}{2 \pi \hbar} e^{-\iu \epsilon V(x_i)/\hbar} e^{-\iu \epsilon (-m^2 \dot{x}_i^2) / 2 m \hbar}  \int_{-\infty}^{\infty} \dd{p^\prime} e^{-\iu \epsilon {p^\prime}^2 / 2 m \hbar} \\
		&= \frac{1}{2 \pi \hbar} e^{\frac{\iu \epsilon}{\hbar} \left(\frac{1}{2} m \dot{x}_i^2 - V(x_i)\right)} \int_{-\infty}^{\infty} \dd{p^\prime} e^{-\iu \epsilon {p^\prime}^2 / 2 m \hbar}
	\end{align}
	Now we use the standard integral
	\begin{equation}
		\int_{-\infty}^{\infty} e^{-\alpha x^2} \dd{x} = \sqrt{\frac{\pi}{\alpha}}
	\end{equation}
	to get
	\begin{equation}
		\int_{-\infty}^{\infty} \dd{p^\prime} e^{-\iu \epsilon p^\prime / 2 m \hbar} = \left(\frac{\pi}{\iu \epsilon/2 m \hbar}\right)^{1/2} = \left(\frac{2 \pi \hbar m}{\iu \epsilon}\right)^{1/2}
	\end{equation}
	Therefore
	\begin{align}
		\braket{x_{i+1} t_{i+1}}{x_i t_i} 
		&= \frac{1}{2 \pi \hbar} \left(\frac{2 \pi \hbar m}{\iu \epsilon}\right)^{1/2} e^{\frac{\iu \epsilon}{\hbar} \left(\frac{1}{2} m \dot{x}_i^2 - V(x_i)\right)} \\
		&= \left(\frac{m}{2\pi\hbar\iu \epsilon}\right)^{1/2} e^{\frac{\iu \epsilon}{\hbar} \left(\frac{1}{2} m \dot{x}_i^2 - V(x_i)\right)}
		\label{chapter12.eqn7-path-integral-propagator}
	\end{align}
	We now substitute equation (\ref{chapter12.eqn7-path-integral-propagator}) into equation (\ref{chapter12.eqn5-propagator-path-integral}) to get
	\begin{align}
		\braket{x t}{x_0 t_0} 
		&= \int \prod_{i=1}^{N-1} \dd{x_i} \prod_{i=0}^{N-1} \braket{x_{i+1} t_{i+1}}{x_i t_i}\\
		&= \int \prod_{i=1}^{N-1} \dd{x_i} \prod_{i=0}^{N-1} \left(\frac{m}{2 \pi \hbar \iu \epsilon}\right)^{1/2} e^{\frac{\iu \epsilon}{\hbar} \left(\frac{1}{2} m \dot{x}_i^2 - V(x_i)\right)} \\
		&= \left(\frac{m}{2 \pi \hbar \iu \epsilon}\right)^{N/2} \int \prod_{i=1}^{N-1} \dd{x_i}e^{\frac{\iu}{\hbar} \sum_{i=0}^{N-1} \epsilon \left(\frac{1}{2} m \dot{x}_i^2 - V(x_i)\right)}
		\label{chapter12.eqn8-path-integral-propagator}
	\end{align}
	We now consider a path $x(t^\prime)$ connecting the initial and the final space-time point such that the value of $x(t^\prime)$ at the intermediate times $t_1, t_2, \ldots, t_{N-1}$ are $x(t_i^\prime)=x_i$. Therefore we can write
	\begin{align}
		\sum_{i=0}^{N-1} \epsilon \left(\frac{1}{2} m \dot{x}_i^2 - V(x_i)\right) 
		&= \int_{t_0}^{t} \left[\frac{1}{2} m \dot{x}^2(t^\prime) - V(x(t^\prime))\right] \dd{t^\prime} \\
		&= \int_{t_0}^{t} L\left(x(t^\prime), \dot{x}(t^\prime)\right) \dd{t^\prime} \\
		&= S[x(t^\prime)]
	\end{align}
	Where $S\left[x(t^\prime)\right]$ is the action calculated along the particular path. Since we are integrating over $x_i\quad (i=1,\ldots,N-1)$, we are effectively summing the exponential in equation (\ref{chapter12.eqn8-path-integral-propagator}) over all conceivable paths connecting $(x_0, t_0)$ to $(x, t)$. We define the path integral as
	\begin{equation}
		\mathcal{D}[x(t^\prime)] = \lim\limits_{N\rightarrow \infty} \left(\frac{m}{2 \pi \hbar \iu \epsilon}\right)^{N/2} \int \prod_{i=1}^{N-1} \dd{x_i}
		\label{chapter12.eqn9-path-integral-propagator}
	\end{equation}
	Therefore, we can write equation (\ref{chapter12.eqn8-path-integral-propagator}) as
	\begin{equation}
		\braket{x, t}{x_0, t_0} = \int \mathcal{D}\left[x(t^\prime)\right] e^{\frac{i}{\hbar} S\left[x(t^\prime)\right]}
		\label{chapter12.eqn10-path-integral-propagator}
	\end{equation}
	This is the path integral formula for the propagator. We can think of equation (\ref{chapter12.eqn10-path-integral-propagator}) as a sysbolic way of writing equation (\ref{chapter12.eqn8-path-integral-propagator}) with $N \rightarrow \infty$. In calculating path integrals we use equation (\ref{chapter12.eqn8-path-integral-propagator}) and set $N \rightarrow \infty$.
	
	\section{Path Integral for a Free Particle}
	For a free particle $V=0$. Therefore the lagrangian is
	\begin{equation}
		L = T - V = T = \frac{1}{2} m \dot{x}^2(t)
		\label{chapter12.eqn11-path-integral-free-particle}
	\end{equation}
	the path integral formula for the propagator of a free particle is
	\begin{align}
		\braket{x, t}{x_0, t_0} 
		&= \int \mathcal{D}\left[x(t^\prime)\right] \exp\left[\frac{\iu}{\hbar} S\left[x(t^\prime)\right]\right] \\
		&= \lim\limits_{N \rightarrow \infty} \left(\frac{m}{2 \pi \hbar \iu \epsilon}\right)^{N/2} \int \prod_{i=1}^{N-1} \dd{x_i} \exp\left[\frac{\iu \epsilon}{\hbar} \sum_{i=0}^{N-1} \frac{1}{2} m \dot{x}_i^2\right]
		\label{chapter12.eqn12-path-integral-free-particle}
	\end{align}
	In equation (\ref{chapter12.eqn12-path-integral-free-particle})
	\begin{equation}
		\epsilon = \frac{t - t_0}{N}
	\end{equation}
	Also $\dot{x}_i$ can be written as
	\begin{equation}
		\dot{x}_i = \frac{x_{i+1} - x_i}{\epsilon}
		\label{chapter12.eqn13-path-integral-free-particle}
	\end{equation}
	
	\begin{figure}
		\centering
		\caption{text}
		\label{chapter12.fig2}
	\end{figure}
	For notational convenience we let $x_N = x$ where $x$ is the final position. We only integrate over the position the particle may have at intermediate times $t_1, t_2, \ldots, t_{N-1}$.\\
	
	Using equation (\ref{chapter12.eqn13-path-integral-free-particle}), equation (\ref{chapter12.eqn12-path-integral-free-particle}) can be written as
	\begin{align}
		\braket{x t}{x_0 t_0} 
		&= \lim\limits_{N \rightarrow \infty} \left(\frac{m}{2\pi \hbar \iu \epsilon}\right)^{N/2} \int \prod_{i=1}^{N-1} \dd{x_i} e^{\frac{\iu \epsilon}{\hbar} \sum_{i=0}^{N-1} \frac{m}{2} \left(\frac{x_{i+1} - x_i}{\epsilon}\right)^2} \\
		&= \lim\limits_{N \rightarrow \infty} \left(\frac{m}{2\pi \hbar \iu \epsilon}\right)^{N/2} \int \prod_{i=1}^{N-1} \dd{x_i} e^{\frac{\iu m}{2\hbar \epsilon} \sum_{i=0}^{N-1} (x_{i+1} - x_i)^2}
		\label{chapter12.eqn14-path-integral-free-particle}
	\end{align}
	At this stage it is convenient to make a change of variable
	\begin{equation}
		y_i = \left(\frac{m}{2\hbar \epsilon}\right)^{1/2} x_i
	\end{equation}
	In terms of new variables equation (\ref{chapter12.eqn14-path-integral-free-particle}) is written as
	\begin{equation}
		\braket{x t}{x_0 t_0} 
		= \lim\limits_{N \rightarrow \infty} \left(\frac{m}{2\pi \hbar \iu \epsilon}\right)^{N/2} \left(\frac{2\hbar \epsilon}{m}\right)^{(N-1)/2} \int \prod_{i=1}^{N-1} \dd{y_i} e^{-\sum_{i=0}^{N-1} \frac{(y_{i+1} - y_i)^2}{i}}
		\label{chapter12.eqn15-path-integral-free-particle}
	\end{equation}
	We now have to do the gaussian integral over the variables $y_1, y_2, \ldots, y_{N-1}$.\\
	
	\textbf{$y_1$ integral}\\
	\begin{equation}
		I_1 = \int_{-\infty}^{\infty} \dd{y_1} \exp\left[-\frac{1}{\iu} \left[(y_1 - y_0)^2 + (y_2 - y_1)^2\right]\right]
	\end{equation}
	consider the exponent
	\begin{equation}
		(y_1 - y_0)^2 + (y_2 - y_1)^2 = 2 y_1^2 - 2 (y_0 + y_2) y_1 + (y_0^2+y_2^2)
	\end{equation}
	therefore,
	\begin{equation}
		I_1 = \exp \left[-\frac{1}{\iu} \left(y_0^2+ y_2^2\right)\right] \int_{-\infty}^{\infty} \dd{y_1} \exp\left[-\frac{1}{\iu} \left(2y_1^2 - 2(y_0 + y_2) y_1\right)
		\right]
	\end{equation}
	Now we use the standard integral
	\begin{equation}
		\int_{-\infty}^{\infty} e^{-\alpha x^2 + \beta x} \dd{x} = \left(\frac{\pi}{\alpha}\right)^{1/2} \exp \left(\frac{\beta^2}{4\alpha}\right)
	\end{equation}
	choosing $\alpha = \frac{2}{\iu}$ and $\beta = \frac{2(y_0 + y_2)}{\iu}$
	\begin{align*}
		I_1 
		&= \exp \left[-\frac{1}{\iu} (y_0^2 + y_2^2)\right] \left(\frac{\iu \pi}{2}\right)^{1/2} \exp\left[\frac{-4 \left(y_0 +y_2\right)^2}{4 (2/\iu)}\right]\\
		&= \exp \left[-\frac{1}{\iu} (y_0^2 + y_2^2)\right] \left(\frac{\iu \pi}{2}\right)^{1/2} \exp\left[\frac{\left(y_0 +y_2\right)^2}{2\iu}\right] \\
		&= \left(\frac{\iu \pi}{2}\right)^{1/2} \exp \left[-\frac{1}{2\iu} \left( 2(y_0^2 + y_2^2) - \left(y_0 +y_2\right)^2 \right)\right]
	\end{align*}
	Thus
	\begin{equation}
		I_1 = \left(\frac{\iu \pi}{2}\right)^{1/2} \exp \left[-\frac{1}{2\iu} \left(y_2 - y_0\right)^2\right]
		\label{chapter12.eqn16-path-integral-free-particle}
	\end{equation}
	Next we do the integral over $y_2$. THe variable $y_2$ occurs in the $i=2$ term in equation (\ref{chapter12.eqn15-path-integral-free-particle}) and also in $I_1$ in equation (\ref{chapter12.eqn16-path-integral-free-particle}). Therefore, the $y_2$ integral is
	\begin{align*}
		I_2 
		&= \int \dd{y_2} \exp \left[-\frac{1}{\iu} \left(y_3 - y_2\right)^2\right] \left(\frac{\iu \pi}{2}\right)^{1/2} \exp \left[-\frac{1}{2 \iu} \left(y_2 - y_0\right)^2\right] \\
		&= \left(\frac{\iu \pi}{2}\right)^{1/2} \int \dd{y_2} \exp \left[-\frac{1}{\iu} \left(y_3 - y_2\right)^2 -\frac{1}{2 \iu} \left(y_2 - y_0\right)^2\right] \\
		&= \left(\frac{\iu \pi}{2}\right)^{1/2} \int \dd{y_2} \exp \left[-\frac{1}{\iu} \{\left(y_3 - y_2\right)^2 + \left(y_2 - y_0\right)^2\}\right] \\
	\end{align*}
	Consider the term in the curly brackets
	\begin{align*}
		2\left(y_3 - y_2\right)^2 + \left(y_2 - y_0\right)^2 &= 2\left(y_3^2 + y_2^2 - 2y_2 y_3\right) + \left(y_2^2 + y_0^2 - 2 y_0 y_2\right) \\
		&= 3 y_2^2 - 2y_2 \left(2y_3 + y_0\right) + \left(2 y_3^2 + y_0^2\right)
	\end{align*}
	where the first term is quadratic in $y_2$, second term is linear in $y_2$ and the last term is independent of $y_2$. We have
	\begin{equation}
		I_2 = \left(\frac{\iu \pi}{2}\right)^{1/2} \exp\left[-\frac{1}{2\iu}\left(2 y_3^2 + y_0^2\right)\right] \int \dd{y} \exp\left[-\frac{1}{2\iu} \left(3 y_2^2 - 2y_2 \left(2 y_3 + y_0\right)\right) \right]
	\end{equation}
	
	We use standard integral (\ref{appendix1.eqn2}) from appendix (\ref{appendix1.Integrals}) and choose
	\begin{align}
		\alpha &= \frac{3}{2\iu} \\
		\beta &= \frac{\left(y_0 2 y_3\right)}{\iu}
	\end{align}
	
	\begin{align}
		I_2 
		&= \left(\frac{\iu \pi}{2}\right)^{1/2} \exp\left[-\frac{1}{2\iu} \left(2 y_3^2 + y_0^2\right)\right] \left(\frac{2\pi \iu}{3}\right)^{1/2} \exp\left[-\frac{\left(y_0 + 2 y_3\right)^2}{6/\iu}\right] \\
		&= \left(\frac{\iu \pi}{2}\right)^{1/2} \left(\frac{2\pi \iu}{3}\right)^{1/2}  \exp\left[-\frac{1}{2\iu} \left(2 y_3^2 + y_0^2\right)\right]  \exp\left[\frac{\left(y_0 + 2 y_3\right)^2}{6\iu}\right] \\
		&= \expval{32}
	\end{align}
	
	\section{Derivation of the propagator for a free particle without using the path integral formula}

\end{enumerate}