\chapter{sheet-4 : Linear Vector Space (continued)}
\section{Adjoint operator}
Consider the equation
\begin{equation}\label{eqn:4.1}
\ket{b} = \hat{K} \ket{a}
\end{equation}
The operator $\hat{K}$ carries the ket $\ket{a}$ to the ket $\ket{b}$. The dual of $\ket{a}$ and $\ket{b}$ are the bras $\bra{a}$ and $\bra{b}$ respectively. Then the operator which carries $\bra{a}$ to $\bra{b}$ is called the adjoint of $\hat{K}$ and is denoted by $\hat{K^\dagger}$. Thus in dual space \ref{eqn:4.1} is
\begin{equation}\label{eqn:4.2}
\bra{b} = \bra{a} \hat{K^\dagger}
\end{equation}
\begin{align}\label{eqn:4.3-4.4}
	\text{ket space \ } \ket{a} _\rightarrow^{\hat{K}} \ket{b}\\
	\text{bra space \ } \bra{a} _\rightarrow^{\hat{K^\dagger}} \bra{b}
\end{align}
Here bra space is the dual of ket space and
\begin{align}\label{eqn:4.5-4.6}
	\ket{a} _\rightarrow^{dc} \bra{a} \\
	\ket{b} _\rightarrow^{dc} \bra{b}
\end{align}
dc $\equiv$ dual correspondence.\\

From equation \ref{eqn:4.1} and \ref{eqn:4.2} it follows that
\begin{equation}\label{eqn:4.7}
\braket{c}{b} = \bra{c}\hat{K}\ket{a}
\end{equation}
and 
\begin{equation}\label{eqn:4.8}
\braket{b}{c} = \bra{a} \hat{K^\dagger} \ket{c}
\end{equation}	 
since
\begin{equation}\label{eqn:4.9}
\braket{b}{c} = \braket{c}{b}^*
\end{equation}
we have
\begin{equation}\label{eqn:4.10}
\bra{c} \hat{K} \ket{a} = \bra{a} \hat{K^\dagger} \ket{c}
\end{equation}
equation \ref{eqn:4.10} is the defining equation for the adjoint $\hat{K^\dagger}$ of the operator $\hat{K}$. In scalar product notation
\begin{equation}\label{eqn:4.11}
\braket{c}{b} \equiv \left(\psi_c, \psi_b \right)
\end{equation}
equation \ref{eqn:4.10} can be written as
\begin{align}\label{eqn:4.12-4.13}
	\left(\psi_c, \hat{K} \psi_a \right) 
	&= \left(\psi_a, \hat{K^\dagger} \psi_c\right)^*\\
	&=	\left(\hat{K^\dagger}\psi_c, \psi_a \right)
\end{align}
In particular, if we take $\ket{c}$ and $\ket{a}$ as the basis states $\ket{i}$ and $\ket{j}$, equation \ref{eqn:4.10} becomes
\begin{align}\label{eqn:4.14-4.18}
	\bra{i}\hat{K}\ket{j} &= \bra{j}\hat{K^\dagger}\ket{i}^* \\
	K_{i j} &= \left(K^\dagger\right)_{j i}^* \\
	K_{j i}^\dagger &= K_{i j}^* \\
	K_{i j}^\dagger &= \left(K_{j i} \right)^* \\
	\left[\hat{K^\dagger} \right] &= \left[K \right]^\dagger
\end{align}
i.e., The matrix representation of the adjoint operator is the hermitian conjugate of the matrix representation of $\hat{K}$.

\section{Hermitian or Self-adjoint operator}
If $\hat{K^\dagger} = \hat{K}$, then $\hat{K}$ is said to be a self-adjoint or a hermitian operator. For a hermitian operator
\begin{equation}\label{eqn:4.19}
\left[K\right] = \left[K^\dagger\right] = \left[K\right]^\dagger
\end{equation}
or
\begin{equation}\label{eqn:4.20}
K_{i j} = K_{j i}^*
\end{equation}
i.e., $\left[K\right]$ is a hermitian matrix.
\begin{enumerate}[label=\textbf{Example \arabic*},start=1]
	\item 
	show that
	\begin{equation}\label{eqn:4.21}
	\left(A B\right)^\dagger = B^\dagger A^\dagger
	\end{equation}
	Answer: \\
	\begin{equation}\label{eqn:4.22}
	\left(\psi_a, \hat{A}\hat{B}\psi_b\right) = \left(\left(  \hat{A}\hat{B}\right)^\dagger\psi_a, \psi_b \right)
	\end{equation}
	Also
	\begin{equation}\label{eqn:4.23}
	\left(\psi_a, \hat{A}\hat{B}\psi_b\right) = \left(  \hat{A}^\dagger\psi_a, \hat{B}\psi_b \right) = \left(  \hat{B}^\dagger\hat{A}^\dagger\psi_a, \psi_b \right)
	\end{equation}
	Comparing these equation (equation \ref{eqn:4.22} and \ref{eqn:4.23}) we get,
	\begin{equation}\label{eqn:4.24}
	\left(\hat{A}\hat{B}\right)^\dagger = \hat{B}^\dagger \hat{A}^\dagger
	\end{equation}
\end{enumerate}
\section{Inverse operator}
An operator $\hat{B}$ is said to be the inverse of $\hat{A}$ if
\begin{equation}\label{eqn:4.25}
\hat{A}\hat{B} = \hat{B}\hat{A} = \hat{\mathbb{I}}
\end{equation}
Obviously, if $\hat{B}$ is the inverse of $\hat{A}$, then $\hat{A}$ is the inverse of $\hat{B}$. We write
\begin{equation}\label{eqn:4.26}
\hat{B} = \hat{A} ^{-1}
\end{equation}
or
\begin{equation}\label{eqn:4.27}
\hat{A} = \hat{B} ^{-1}
\end{equation}
If equation \ref{eqn:4.25} is satisfied.

\section{Unitary operator}
An operator $\hat{U}$ is said to be unitary if
\begin{equation}
\hat{U} \hat{U^\dagger} = \hat{U^\dagger} \hat{U} = \hat{\mathbb{I}}
\end{equation}
i.e., if
\begin{equation}
\hat{U^\dagger} = \hat{U}^{-1}
\end{equation}
Thus, for a unitary operator, its adjoint is $@@@@@@@@@@@@@@@@$

\section{Function of Operators}

